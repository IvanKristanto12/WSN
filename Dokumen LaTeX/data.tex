%_____________________________________________________________________________
%=============================================================================
% data.tex v10 (22-01-2017) dibuat oleh Lionov - T. Informatika FTIS UNPAR
%
% Perubahan pada versi 10 (22-01-2017)
%	- Penambahan overfullrule untuk memeriksa warning
%  	- perubahan mode buku menjadi 4: bimbingan, sidang(1), sidang akhir dan 
%     buku final
%	- perbaikan perintah pada beberapa bagian
%  	- perubahan pengisian tulisan "daftar isi" yang error
%  	- penghilangan lipsum dari file ini
%_____________________________________________________________________________
%=============================================================================

%=============================================================================
% 								PETUNJUK
%=============================================================================
% Ini adalah file data (data.tex)
% Masukkan ke dalam file ini, data-data yang diperlukan oleh template ini
% Cara memasukkan data dijelaskan di setiap bagian
% Data yang WAJIB dan HARUS diisi dengan baik dan benar adalah SELURUHNYA !!
% Hilangkan tanda << dan >> jika anda menemukannya
%=============================================================================

%_____________________________________________________________________________
%=============================================================================
% 								BAGIAN 0
%=============================================================================
% Entri untuk memperbaiki posisi "DAFTAR ISI" jika tidak berada di bagian 
% tengah halaman. Sayangnya setiap sistem menghasilkan posisi yang berbeda.
% Isilah dengan 0 atau 1 (e.g. \daftarIsiError{1}). 
% Pemilihan 0 atau 1 silahkan disesuaikan dengan hasil PDF yang dihasilkan.
%=============================================================================
\daftarIsiError{0}   
%\daftarIsiError{1}   
%=============================================================================

%_____________________________________________________________________________
%=============================================================================
% 								BAGIAN I
%=============================================================================
% Tambahkan package2 lain yang anda butuhkan di sini
%=============================================================================
\usepackage{booktabs} 
\usepackage{longtable}
\usepackage{amssymb}
\usepackage{todo}
\usepackage{verbatim} 		%multiline comment
\usepackage{pgfplots}
\usepackage{amsmath}
\usepackage{mathtools}
\usepackage{program}

\usepackage{algpseudocode}
\usepackage{algorithm}
\usepackage{amsmath}

\usepackage{dirtree}% untuk dirtree
\usepackage{gensymb}
\usepackage{csvsimple}
\usepackage{multirow}
\usepackage{listings} 
\usepackage{longtable}

\usepackage[bottom]{footmisc}


\lstset{numbers=left,stepnumber=1, numbersep=5pt, frame=leftline,
	tabsize=4, breaklines=true, basicstyle=\fontfamily{fvm}\selectfont\tiny, 
	commentstyle=\itshape\color{gray}, keywordstyle=\bfseries\color{blue}, 
	identifierstyle=\color{black}, stringstyle=\color{orange},
	literate={-}{-}1{-\,-}{--}1
} 

%\overfullrule=3mm % memperlihatkan overfull 
%=============================================================================

%_____________________________________________________________________________
%=============================================================================
% 								BAGIAN II
%=============================================================================
% Mode dokumen: menetukan halaman depan dari dokumen, apakah harus mengandung 
% prakata/pernyataan/abstrak dll (termasuk daftar gambar/tabel/isi) ?
% - final 		: hanya untuk buku skripsi, dicetak lengkap: judul ina/eng, 
%   			  pengesahan, pernyataan, abstrak ina/eng, untuk, kata 
%				  pengantar, daftar isi (daftar tabel dan gambar tetap 
%				  opsional dan dapat diatur), seluruh bab dan lampiran.
%				  Otomatis tidak ada nomor baris dan singlespacing
% - sidangakhir	: buku sidang akhir = buku final - (pengesahan + pernyataan +
%   			  untuk + kata pengantar)
%				  Otomatis ada nomor baris dan onehalfspacing 
% - sidang 		: untuk sidang 1, buku sidang = buku sidang akhir - (judul 
%				  eng + abstrak ina/eng)
%				  Otomatis ada nomor baris dan onehalfspacing
% - bimbingan	: untuk keperluan bimbingan, hanya terdapat bab dan lampiran
%   			  saja, bab dan lampiran yang hendak dicetak dapat ditentukan 
%				  sendiri (nomor baris dan spacing dapat diatur sendiri)
% Mode default adalah 'template' yang menghasilkan isian berwarna merah, 
% aktifkan salah satu mode di bawah ini :
%=============================================================================
%\mode{bimbingan} 		% untuk keperluan bimbingan
%\mode{sidang} 			% untuk sidang 1
%\mode{sidangakhir} 	% untuk sidang 2 / sidang pada Skripsi 2(IF)
\mode{final} 			% untuk mencetak buku skripsi 
%=============================================================================

%_____________________________________________________________________________
%=============================================================================
% 								BAGIAN III
%=============================================================================
% Line numbering: penomoran setiap baris, nomor baris otomatis di-reset ke 1
% setiap berganti halaman.
% Sudah dikonfigurasi otomatis untuk mode final (tidak ada), mode sidang (ada)
% dan mode sidangakhir (ada).
% Untuk mode bimbingan, defaultnya ada (\linenumber{yes}), jika ingin 
% dihilangkan, isi dengan "no" (i.e.: \linenumber{no})
% Catatan:
% - jika nomor baris tidak kembali ke 1 di halaman berikutnya, compile kembali
%   dokumen latex anda
% - bagian ini hanya bisa diatur di mode bimbingan
%=============================================================================
\linenumber{no} 
%\linenumber{yes}
%=============================================================================

%_____________________________________________________________________________
%=============================================================================
% 								BAGIAN IV
%=============================================================================
% Linespacing: jarak antara baris 
% - single	: otomatis jika ingin mencetak buku skripsi, opsi yang 
%			     disediakan untuk bimbingan, jika pembimbing tidak keberatan 
%			     (untuk menghemat kertas)
% - onehalf	: otomatis jika ingin mencetak dokumen untuk sidang
% - double 	: jarak yang lebih lebar lagi, jika pembimbing berniat memberi 
%             catatan yg banyak di antara baris (dianjurkan untuk bimbingan)
% Catatan: bagian ini hanya bisa diatur di mode bimbingan
%=============================================================================
\linespacing{single}
%\linespacing{onehalf}
%\linespacing{double}
%=============================================================================

%_____________________________________________________________________________
%=============================================================================
% 								BAGIAN V
%=============================================================================
% Tidak semua skripsi memuat gambar dan/atau tabel. Untuk skripsi yang tidak 
% memiliki gambar dan/atau tabel, maka tidak diperlukan Daftar Gambar dan/atau 
% Daftar Tabel. Sayangnya hal tsb sulit dilakukan secara manual karena 
% membutuhkan kedisiplinan pengguna template.  
% Jika tidak ingin menampilkan Daftar Gambar dan/atau Daftar Tabel, karena 
% tidak ada gambar atau tabel atau karena dokumen dicetak hanya untuk 
% bimbingan, isi dengan "no" (e.g. \gambar{no})
%=============================================================================
\gambar{yes}
%\gambar{no}
\tabel{yes}
%\tabel{no}  
%\kode{yes}
\kode{no} 
%\notasi{yes}
\notasi{no}
%=============================================================================

%_____________________________________________________________________________
%=============================================================================
% 								BAGIAN VI
%=============================================================================
% Pada mode final, sidang da sidangkahir, seluruh bab yang ada di folder "Bab"
% dengan nama file bab1.tex, bab2.tex s.d. bab9.tex akan dicetak terurut, 
% apapun isi dari perintah \bab.
% Pada mode bimbingan, jika ingin:
% - mencetak seluruh bab, isi dengan 'all' (i.e. \bab{all})
% - mencetak beberapa bab saja, isi dengan angka, pisahkan dengan ',' 
%   dan bab akan dicetak terurut sesuai urutan penulisan (e.g. \bab{1,3,2}). 
% Catatan: Jika ingin menambahkan bab ke-3 s.d. ke-9, tambahkan file 
% bab3.tex, bab4.tex, dst di folder "Bab". Untuk bab ke-10 dan 
% seterusnya, harus dilakukan secara manual dengan mengubah file skripsi.tex
% Catatan: bagian ini hanya bisa diatur di mode bimbingan
%=============================================================================
\bab{all}
%=============================================================================

%_____________________________________________________________________________
%=============================================================================
% 								BAGIAN VII
%=============================================================================
% Pada mode final, sidang dan sidangkhir, seluruh lampiran yang ada di folder 
% "Lampiran" dengan nama file lampA.tex, lampB.tex s.d. lampJ.tex akan dicetak 
% terurut, apapun isi dari perintah \lampiran.
% Pada mode bimbingan, jika ingin:
% - mencetak seluruh lampiran, isi dengan 'all' (i.e. \lampiran{all})
% - mencetak beberapa lampiran saja, isi dengan huruf, pisahkan dengan ',' 
%   dan lampiran akan dicetak terurut sesuai urutan (e.g. \lampiran{A,E,C}). 
% - tidak mencetak lampiran apapun, isi dengan "none" (i.e. \lampiran{none})
% Catatan: Jika ingin menambahkan lampiran ke-C s.d. ke-I, tambahkan file 
% lampC.tex, lampD.tex, dst di folder Lampiran. Untuk lampiran ke-J dan 
% seterusnya, harus dilakukan secara manual dengan mengubah file skripsi.tex
% Catatan: bagian ini hanya bisa diatur di mode bimbingan
%=============================================================================
\lampiran{all}
%=============================================================================

%_____________________________________________________________________________
%=============================================================================
% 								BAGIAN VIII
%=============================================================================
% Data diri dan skripsi/tugas akhir
% - namanpm		: Nama dan NPM anda, penggunaan huruf besar untuk nama harus 
%				  benar dan gunakan 10 digit npm UNPAR, PASTIKAN BAHWA 
%				  BENAR !!! (e.g. \namanpm{Jane Doe}{1992710001}
% - judul 		: Dalam bahasa Indonesia, perhatikan penggunaan huruf besar, 
%				  judul tidak menggunakan huruf besar seluruhnya !!! 
% - tanggal 	: isi dengan {tangga}{bulan}{tahun} dalam angka numerik, 
%				  jangan menuliskan kata (e.g. AGUSTUS) dalam isian bulan.
%			  	  Tanggal ini adalah tanggal dimana anda akan melaksanakan 
%				  sidang ujian akhir skripsi/tugas akhir
% - pembimbing	: pembimbing anda, lihat daftar dosen di file dosen.tex
%				  jika pembimbing hanya 1, kosongkan parameter kedua 
%				  (e.g. \pembimbing{\JND}{} ), \JND adalah kode dosen
% - penguji 	: para penguji anda, lihat daftar dosen di file dosen.tex
%				  (e.g. \penguji{\JHD}{\JCD} )
% !!Lihat singkatan pembimbing dan penguji anda di file dosen.tex!!
% Petunjuk: hilangkan tanda << & >>, dan isi sesuai dengan data anda
%=============================================================================
\namanpm{Ivan Kristanto}{2016730082}
\tanggal{9}{6}{2020}
\pembimbing{\ELH}{}    
\penguji{\MTA}{\LCA}
%=============================================================================

%_____________________________________________________________________________
%=============================================================================
% 								BAGIAN IX
%=============================================================================
% Judul dan title : judul bhs indonesia dan inggris
% - judulINA: judul dalam bahasa indonesia
% - judulENG: title in english
% Petunjuk: 
% - hilangkan tanda << & >>, dan isi sesuai dengan data anda
% - langsung mulai setelah '{' awal, jangan mulai menulis di baris bawahnya
% - gunakan \texorpdfstring{\\}{} untuk pindah ke baris baru
% - judul TIDAK ditulis dengan menggunakan huruf besar seluruhnya !!
%=============================================================================
\judulINA{Pengembangan Aplikasi Ekstraksi Fitur Domain
Waktu/Frekuensi untuk Data Akselerometer di WSN}
\judulENG{Time / Frequency Domain Feature Extraction Application Development for Accelerometer Data at WSN}
%_____________________________________________________________________________
%=============================================================================
% 								BAGIAN X
%=============================================================================
% Abstrak dan abstract : abstrak bhs indonesia dan inggris
% - abstrakINA: abstrak bahasa indonesia
% - abstrakENG: abstract in english 
% Petunjuk: 
% - hilangkan tanda << & >>, dan isi sesuai dengan data anda
% - langsung mulai setelah '{' awal, jangan mulai menulis di baris bawahnya
%=============================================================================
\abstrakINA{Wireless Sensor Network (WSN) adalah jaringan nirkabel berupa node - node sensor yang memiliki kemampuan melakukan komputasi, komunikasi dengan node lain, dan {\it sensing}. Salah satu sensor yang terdapat pada sensor node adalah sensor akselerometer yang berfungsi sebagai pengukur getaran. Sensor akselerometer mengukur akselerasi pada 3 sumbu ($x$,$y$,$z$). Hasil dari {\it sensing} akselerometer belum berupa informasi dan dapat diolah dengan melakukan ekstraksi fitur domain waktu ke frekuensi. Dari hasil ekstraksi fitur dari data {\it sensing} akselerometer dapat digunakan sebagai deteksi getaran dan pengambilan keputusan untuk pemrosesan lanjut.

Ada beberapa teknik ekstraksi fitur domain waktu ke frekuensi. Salah satu teknik ekstraksi fitur yang ada adalah STFT ({\it Short Time Fourier Transform}). STFT merupakan pengembangan dari teknik ekstraksi fitur FFT ({\it Fast Fourier Transform}). Ekstrasi fitur dilakukan pada sensor node sehingga hasil {\it sensing} node merupakan hasil dari STFT.

Pada skripsi ini dibuat aplikasi untuk melakukan ekstraksi fitur untuk data akselerometer. Aplikasi ini mengatur interkasi komputer pengguna, {\it base station}, dan sensor node. Algoritma ekstraksi fitur yang digunakan adalah STFT. Proses ekstraksi fitur data akselerometer dilakukan di sensor node dan hasil ekstraksi fitur dikirim ke {\it base station} dan ditampilkan di komputer.  

Hasil dari pengujian menunjukkan bahwa aplikasi ekstraksi fitur ini dapat dibangun di WSN. Hasil dari ekstraksi fitur bergantung pada letak sensor melakukan {\it sensing} dan tidak bergantung dengan topologi WSN yang digunakan. 
 } 
 
\abstrakENG{Wireless Sensor Network (WSN) is a wireless network in the form of sensor nodes that have the ability to compute, communicate with other nodes, and sensing. One of the sensors contained in the node is the Accelerometer sensor that measures vibration from acceleration. The Accelerometer sensor measures acceleration on 3 axes ($ x $, $ y $, $ z $). The results of the sensing are not in the form of information and can be processed by time domain frequency extracting  features. From the results of feature extraction from the accelerometer data, it can be used as vibration detection and decision making for further processing.

There are several time domain frequency feature extraction techniques. One of feature extraction techniques available is STFT (Short Time Fourier Transform). STFT is a development of FFT (Fast Fourier Transform). Feature extraction is done on the sensor node
so the results of {\ it sensing} node are the result of STFT.

In this thesis is created an application that can extract accelerometer data feature. This application controls user computer, base station, and sensor node interactions. STFT feature extraction algorithm is used for the extraction process. The process is done in sensor node and the result is sent to base station and displayed in user computer.

The results show that this application can be built on WSN. The feature extraction result depend on where the sensor perform sensing and does not 
depend on the WSN topology used. } 
%=============================================================================

%_____________________________________________________________________________
%=============================================================================
% 								BAGIAN XI
%=============================================================================
% Kata-kata kunci dan keywords : diletakkan di bawah abstrak (ina dan eng)
% - kunciINA: kata-kata kunci dalam bahasa indonesia
% - kunciENG: keywords in english
% Petunjuk: hilangkan tanda << & >>, dan isi sesuai dengan data anda.
%=============================================================================
\kunciINA{sensor jaringan nirkabel, akselerometer, ekstraksi fitur, transformasi fourier}
\kunciENG{wireless sensor network, accelerometer, feature extraction, fourier transform}
%=============================================================================

%_____________________________________________________________________________
%=============================================================================
% 								BAGIAN XII
%=============================================================================
% Persembahan : kepada siapa anda mempersembahkan skripsi ini ...
% Petunjuk: hilangkan tanda << & >>, dan isi sesuai dengan data anda.
%=============================================================================
\untuk{Dipersembahkan kepada Tuhan, keluarga, saudara, dan teman - teman yang telah mendukung}
%=============================================================================

%_____________________________________________________________________________
%=============================================================================
% 								BAGIAN XIII
%=============================================================================
% Kata Pengantar: tempat anda menuliskan kata pengantar dan ucapan terima 
% kasih kepada yang telah membantu anda bla bla bla ....  
% Petunjuk: hilangkan tanda << & >>, dan isi sesuai dengan data anda.
%=============================================================================
\prakata{
Puji syukur penulis panjatkan kepada Tuhan Yang Maha Esa karena atas karunia-Nya, penulis
dapat menyelesaikan penyusunan skripsi yang berjudul "Pengembangan Aplikasi Ekstraksi Fitur Domain
Waktu/Frekuensi untuk Data Akselerometer di WSN". Selama penyusunan skripsi ini, penulis menghadapi banyak kendala dan berbagai masalah. Penulis menyadari bahwa penyusunan skripsi ini juga tidak terlepas dari bantuan berbagai pihak, baik
langsung maupun tidak langsung. Secara khusus, penulis ingin berterima kasih kepada:
\begin{enumerate}
\item Tuhan Yesus atas Anugrah, Berkat, dan Rahmat-Nya.
\item Keluarga yang selalu memberikan dukungan kepada penulis baik berupa doa atau dukungan
mental serta materiil.
\item Bapa \ELH $ $ selaku dosen pembimbing yang telah membimbing penulis dan memberikan dukungan maupun bantuan kepada penulis dalam proses penyusunan skripsi ini.
\item Ibu \MTA $ $ dan Ibu \LCA $ $ selaku dosen penguji yang telah
memberikan kritik dan saran yang membangun sehingga penelitian ini menjadi lebih baik.
\item Teman-teman sejurusan Teknik Informatika UNPAR angkatan 2016 dan teman - teman di luar perkuliahan yang telah
menemani penulis dalam menyelesaikan perkuliahan dari awal semester sampai akhir semester.
\item Teman seperjuangan skripsi yang berdosen pembimbing sama dengan penulis, bimbingan bersama, saling membantu, dan saling
memberikan dukungan satu sama lain selama menyusun skripsi ini.
\end{enumerate}

Penulis menyadari bahwa penelitian ini masih jauh dari kata sempurna. Oleh karena itu, penulis
memohon maaf jika terdapat kesalahan. Penulis juga mengharapkan kritik dan saran yang memba-
ngun untuk menyempurnakan penelitian ini. Semoga penelitian ini dapat memberi informasi yang
bermanfaat dan menjadi inspirasi untuk penelitian-penelitian berikutnya.} 
%=============================================================================

%_____________________________________________________________________________
%=============================================================================
% 								BAGIAN XIV
%=============================================================================
% Jenis tandatangan di lembar pernyataan mahasiswa tentang plagiarisme.
% Ada 4 pilihan:
%   - digital   : diisi menggunakan digital signature (menggunakan pengolah
%                 pdf seperti Adobe Acrobat Reader DC).
%   - gambar    : diisi dengan gambar tandatangan mahasiswa (file tandatangan
%                 bertipe pdf/png/jpg). Dianjukan menggunakan warna biru.
%                 Letakkan gambar di folder "Gambar" dengan nama ttd.jpg/
%                 ttd.png/ttd.pdf (tergantung jenis file. Hapus file ttd.jpg
%                 yang digunakan sebagai contoh
%   - materai   : khusus bagi yang ingin mencetak buku dan menandatangani di 
%                 atas materai. Sama dengan pilihan ``digital'' dan dicetak.
%   - kosong    : tempat kosong ini bisa diisi dengan tanda tangan yang
%                 digambar langsung di atas pdf (fill&sign via acrobat, tanda
%                 tangan dapat dibuat dengan mouse atau stylus)
%=============================================================================
%\ttd{digital}
%\ttd{gambar}
%\ttd{materai}
%\ttd{kosong}
%=============================================================================
\ttd{gambar}

%_____________________________________________________________________________
%=============================================================================
% 								BAGIAN XV
%=============================================================================
% Pilihan tanda tangan digital untuk dosen/pejabat:
%   - no    : pdf TIDAK dapat ditandatangani secara digital, mengakomodasi 
%             yang akan menandatangani via ``menulis'' di file pdf
%   - yes   : pdf dapat ditandatangani secara digital
% 
% PERHATIAN: perubahan ini harus ditanyakan ke kaprodi/dosen koordinator, 
% apakah harus mengisi ``no" atau ``yes". Default = no 
%=============================================================================
%\ttddosen{yes}
%=============================================================================
\ttddosen{no}

%_____________________________________________________________________________
%=============================================================================
% 								BAGIAN XVI
%=============================================================================
% Tambahkan hyphen (pemenggalan kata) yang anda butuhkan di sini 
%=============================================================================
\hyphenation{ma-te-ma-ti-ka}
\hyphenation{fi-si-ka}
\hyphenation{tek-nik}
\hyphenation{in-for-ma-ti-ka}
%=============================================================================

%_____________________________________________________________________________
%=============================================================================
% 								BAGIAN XVII
%=============================================================================
% Tambahkan perintah yang anda buat sendiri di sini 
\renewcommand{\vtemplateauthor}{lionov}
\pgfplotsset{compat=newest}

\algblock[TryCatchFinally]{try}{endtry}
\algcblock[TryCatchFinally]{TryCatchFinally}{finally}{endtry}
\algcblockdefx[TryCatchFinally]{TryCatchFinally}{catch}{endtry}
[1]{\textbf{catch} #1}
{\textbf{end try}}

\algdef{SE}[DOWHILE]{Do}{doWhile}{\algorithmicdo}[1]{\algorithmicwhile\ #1}

\algdef{SE}[SUBALG]{Indent}{EndIndent}{}{\algorithmicend\ }
\algtext*{Indent}
\algtext*{EndIndent}

%=============================================================================
\renewcommand{\vtemplateauthor}{lionov}
\pgfplotsset{compat=newest}
%=============================================================================
